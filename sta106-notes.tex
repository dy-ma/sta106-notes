\documentclass[10pt]{article}

\usepackage[utf8]{inputenc}
\usepackage[margin=0.5in, a4paper, landscape]{geometry}
\usepackage{amsmath}
\usepackage{amssymb}
\usepackage{blindtext}
\usepackage{multicol}
\usepackage[x11names]{xcolor}
\usepackage[hidelinks]{hyperref}

\setlength{\parindent}{0pt}

\newcommand{\define}[1]{\colorbox{Thistle2}{#1}}
\newcommand{\emphas}[1]{\colorbox{DarkSeaGreen2}{#1}}

\newcommand{\mean}[1]{\mu_{#1}}
\renewcommand{\exp}[1]{E\left\{#1\right\}} % expected
\newcommand{\var}[1]{\sigma^2\left\{#1\right\}}
\newcommand{\dev}[1]{\sigma\left\{#1\right\}}
\newcommand{\sigvar}[1]{\sigma^2_{#1}} % sigma subscript versions
\newcommand{\sigdev}[1]{\sigma_{#1}}
\newcommand{\svar}[1]{s^2_{#1}} % sigma subscript versions
\newcommand{\sdev}[1]{s_{#1}}
\newcommand{\err}[1]{\epsilon_{#1}}

\newcommand{\prob}[1]{P\left\{#1\right\}}

\newcommand{\drawline}{\noindent\rule{\linewidth}{0.1pt}}

\begin{document}
\begin{multicols}{3}

    \fbox{\parbox{2.8in}{
            \textbf{STA 106 Notes - M. Pouokam}\\
            \textbf{Dylan M Ang}\\
            \textbf{\today}
        }}


    \tableofcontents

    \section{Terminology}

    \define{Subject}: A person, place or thing from which we measure data.

    \define{Population}: The collection of all subjects of interest.

    \define{Sample}: A subset of the population, from which we collect data.

    \define{Response/Dependent Variable}: The variable which is of primary interest.

    \define{Explanatory/Independent Variable}: The variable which we believe helps explain some of the variance in the response variable.

    For this class, the response variable is \emphas{numerical}, and the explanatory variable/s are \emphas{categorical}. This is often phrased as ``how does this numerical variable differ by group?''

    \define{Random Variable}: A variable whose outcome is random. These are typically denoted by capital letters.

    \section{Notation}

    Let $Y =$ the random variable denoting all possible values of the response variable.

    Let $y =$ an observed value of $Y$ (in other words, measured observations).

    Let $Y_{ij}=$ the rv denoting all possible values of the $jth$ observation in group $i$.

    Let $y_{ij}=$ the $ith$ observed value of the $jth$ group in $Y$.

    As an example, let $i=1$ denote sex M, and $i=2$ denote sex F.

    \begin{itemize}
        \item $Y_{13}=$ All possible values of height for the 3rd male. The outcome is random and unknown.
        \item $y_{13}= 72$ inches. The specific observed value of the 3rd male once measured.
    \end{itemize}

    $\mu_i=$ The population mean for group i (a single value).

    $\bar Y_i =$ All possible values of the sample mean for group i.

    $\bar y_i =$ A specific, observed value of $\bar Y_i$. In other words, the mean of one given sample.

    $\sigma_i =$ The population standard deviation for group i.

    $S_i=$ All possible values of the sample standard deviation.

    $s_i=$ A specific, observed value of $S_i$. In other words, the standard deviation of one given sample.

    The book does not make this distinction $\implies Y=y$ and $\bar Y = \bar y$

    \define{Parameter}: The unknown population value of some statistic. For example $\mu_i, \sigma_i$. These values are constant (non-random) if we could measure the population we would know the true value.

    The goal of statistics 106 is to estimate parameters with sample values, and use the assumed distribution of those sample values to form \emphas{Hypothesie Tests (HTs)} and \emphas{Confidence Intervals (CIs)}.

    \section{Mean and Variance of RVs}

    Let $Y_i$ be drawn from a distribution with population mean $\mu_Y$ and population standard deviation $\sigma_Y$.

    Let the \define{mean} of $Y_i = \mean{Y_i} = \exp{Y_i} = \mean{Y}$.

    Let the \define{standard deviation} of $Y_i = \sigdev{Y_i} = \dev{Y_i} = \sigdev{Y}$.

    \subsection{Linear Combinations of RVs}

    Let $Y^*$ be a linear combination of $Y$ where $a,b \in \mathbb{R}$, then

    \begin{center}
        \begin{tabular}{l | l | l}
            Combination    & Mean                         & Variance                        \\ \hline
            $Y^* = a + Y$  & $\mean{Y^*} = a + \mean{Y}$  & $\sigvar{Y^*} = \sigvar{Y}$     \\
            $Y^* = bY$     & $\mean{Y^*} = b\mean{Y}$     & $\sigvar{Y^*} = b^2 \sigvar{Y}$ \\
            $Y^* = a + bY$ & $\mean{Y^*} = a + b\mean{Y}$ & $\sigvar{Y^*} = b^2 \sigvar{Y}$
        \end{tabular}
    \end{center}

    \subsection{Summation Identities}

    Let $Y_1, Y_2, \dots, Y_n$ be RVs
    \begin{align*}
        \exp{\sum^n_{i=1} Y_i} & = \exp{Y_1 + Y_2 + \dots + Y_n}           \\
                               & = \exp{Y_1} + \exp{Y_2} + \dots \exp{Y_n} \\
                               & = \sum^n_{i=1} \exp{Y_i}
    \end{align*}

    If $Y_1, Y_2, \dots, Y_n$ are independent RVs,
    \begin{align*}
        \var{\sum^n_{i=1} Y_i} & = \sum^n_{i=1} \var{Y_i}
    \end{align*}

    Let $\bar Y = \frac{1}{n} \sum^n_{i=1} Y_i$, $Y_i$ is independent with mean $\mean{Y}$ and std.dev. $\sigdev{Y}$.
    \begin{align*}
        \exp{\bar Y} & = \exp{\frac{1}{n} \sum^n_{i=1} Y_i} = \exp{\frac{1}{n} (Y_1 + Y_2 + \dots + Y_n)}       \\
                     & = \frac{1}{n} \exp{Y_1 + Y_2 + \dots + Y_n}                                              \\
                     & = \frac{1}{n} \Big(\exp{Y_1} + \exp{Y_2} + \dots + \exp{Y_n}\Big)                        \\
                     & = \frac{1}{n} \sum^n_{i=1} \exp{Y_i} = \frac{1}{n} \sum^n_{i=1} \mean{Y}                 \\
                     & = \frac{1}{n} (\mean{Y} + \mean{Y} + \dots + \mean{Y}) = \frac{1}{n} (n*\mean{Y})        \\
        \exp{\bar Y} & = \mean{Y}                                                                               \\
        \var{\bar Y} & = \var{\frac{1}{n} \sum^n_{i=1} Y_i} = \left(\frac{1}{n}\right)^2 \var{\sum^n_{i=1} Y_i} \\
                     & = \left(\frac{1}{n}\right)^2 \sum^n_{i=1} \var{Y_i}                                      \\
    \end{align*}

    \begin{align*}
        \mean{(\bar{Y_1} - \bar{Y_2})} & = \mean{1} - \mean{2} \\
        \mean{(\bar{Y_1} + \bar{Y_2})} & = \mean{1} + \mean{2} \\
        \var{(\bar{Y_1} - \bar{Y_2})} & = \var{1} + \var{2} \\
        \var{(\bar{Y_1} + \bar{Y_2})} & = \var{1} + \var{2} 
    \end{align*}

    \section{Normal RVs and \texorpdfstring{$\chi^2$}{X2} RV} % texorpdfstring allows for tokens for bookmarks

    \subsection{Normal RVs}

    A normal RV follows a bell curve created by a \emphas{probability density function (pdf)}.

    If $Y$ is normally distributed with mean $\mean{Y}$ and std dev $\sigdev{Y}$, we say that $Y \sim N(\mean{Y}, \sigdev{Y})$

    $Y \sim N(\mean{Y}, \sigdev{Y}) \implies Y^* = a + bY \sim N(a + b\mean{Y}, b\sigdev{Y})$

    From this we can get two more results,

    If $Y_1, \dots, Y_n$ independent and $Y_i \sim N(\mean{Y}, \sigdev{Y})$, then

    \begin{enumerate}
        \item $\bar Y \sim N(\mean{Y}, \sigdev{Y}/\sqrt{n})$
        \item $\sum Y_i \sim N(n \mean{Y}, \sqrt{n \sigdev{Y}})$
    \end{enumerate}

    The \define{standard normal distribution} is a specific linear combination of a general normal distribution, denoted $Z$.

    Let $Y \sim N(\mean{Y}, \sigdev{Y})$
    \begin{align*}
        Z          & = \frac{Y - \mean{Y}}{\sigdev{Y}} = \frac{Y}{\sigdev{Y}} - \frac{\mean{Y}}{\sigdev{Y}} \\
        \exp{Z}    & = \frac{-\mean{Y}}{\sigdev{Y}} + \mean{Y} (\frac{1}{\sigdev{Y}}) = 0                   \\
        \sigvar{Z} & = \left(\frac{1}{\sigdev{Y}}\right)^2 \sigvar{Y} = 1                                   \\
    \end{align*}

    Therefore $Z \sim N(0,1)$

    \subsection{\texorpdfstring{$\chi^2$}{chi2} Distribution}

    The \define{$\chi^2$ distribution} (chi-squared) is a sum of independent squared Z distributions.

    Let $Z_1, Z_2, \dots, Z_n$ be independent RVs where $Z_i \sim N(0,1)$

    \begin{align*}
        X              & = Z_1^2 + \dots + Z_n^2 \sim \chi_r^2 \text{ with degrees of freedom} \\
        r              & = \text{ The number of summed and squared } Z_i^2                     \\
        \exp{\chi_r^2} & = r
    \end{align*}

    \section{Hypothesis Testing and Confidence Intervals}

    \subsection{Testing for difference in means}

    Step 1: Declare Hypothesis
    \begin{align*}
         & H_0: \mean{1} = \mean{2} \;or\; \mean{1} \leq \mean{2} \;or\; \mean{1} \geq \mean{2} \\
         & H_A: \mean{1} \neq \mean{2} \;or\; \mean{1} > \mean{2} \;or\; \mean{1} < \mean{2}
    \end{align*}

    Step 2: Calculate test-statistic
    \begin{align*}
        t_s & = \frac{(\bar{y_1} - \bar{y_2}) - \Delta_0}{\sqrt{\left(\frac{\svar{1}}{n_1} + \frac{\svar{2}}{n_2}\right)}} \\
    \end{align*}

    If equal variances are assumed, the following test statistic formula can be used.
    \begin{align*}
        t_s      & = \frac{(\bar{y_1} - \bar{y_2}) - \Delta_0}{\sqrt{\svar{p} (\frac{1}{n_1} + \frac{1}{n_2})}} \sim t, df = n_1 + n_2 - 2 \\
        \svar{p} & = \frac{\svar{1} (n_1-1) + \svar{2} (n_2-1)}{n_1 + n_2 - 2}                                                             \\
    \end{align*}

    \define{$t_s$} = The number of estimated standard deviations our sample difference in means is from the null.

    Step 3: Calculate the p-value

    If $H_A \implies$ p-value

    \begin{align*}
        H_A                    & \implies & p                 \\
        \mean{1} \neq \mean{2} &          & 2\prob{t > |t_s|} \\
        \mean{1} < \mean{2}    &          & \prob{t < t_s}    \\
        \mean{1} > \mean{2}    &          & \prob{t > t_s}
    \end{align*}

    p-value $= \prob{\text{our data or more extreme} | H_0 \; TRUE}$

    \define{p-value} = probability of observing our sample data or more extreme, if in reality the null hypothesis were true.

    Step 4: State decision rule and conclusion

    \begin{align*}
        If \; p-value & < \alpha, reject \; H_0                \\
        If \; p-value & \geq \alpha, fail \;to\; reject \; H_0 \\
    \end{align*}

    Recall that
    \begin{align*}
        \alpha & = \prob{\text{Type I Error}} = \prob{reject \; H_0 | H_0 true} \\
    \end{align*}

    \subsection{Confidence Interval for difference in means}

    The corresponding $(1-\alpha)100\%$ CI for $(\mean{1} - \mean{2})$ is

    \begin{align*}
        (\bar{y_1} - \bar{y_2}) & \pm t_{1 - \alpha/2; n_1 + n_2 - 2} \sqrt{\svar{p}\left(\frac{1}{n_1} + \frac{1}{n_2}\right)} \\
    \end{align*}
    $t_{1 - \alpha/2; n_1 + n_2 - 2}$ is the $(1-\alpha)100th$ percentile of a t distribution with $df = n_1 + n_2 - 2$

    \subsection{Assumptions}
    \begin{enumerate}
        \item Random samples from both groups.
        \item Groups are independent
        \item $\sigdev{1} = \sigdev{2}$ if using $s_p$ formula.
        \item $\bar{Y_1} - \bar{Y_2}$ is distributed normally, either because
              \begin{enumerate}
                  \item Both populations are normal
                  \item $n_1$ and $n_2 \geq 30$ (Central limit theorem)
              \end{enumerate}
    \end{enumerate}

    \section{Experimental Design}

    \subsection{Sampling}

    In ANOVA studies, the sampling scheme is very important. Typically, the \emphas{categorical variable} is seen as a \emphas{treatment}, and the goal is to see if it had an effect on the numerical variable.

    In an \define{experiment}, subjects and randomly assigned a \emphas{treatment}, and the results are assessed to find a causal relationship between variables.

    In an \define{observational study}, subjects are randomly sampled and may fall into natural \emphas{treatment groups}, but are not assigned one. The data is assessed to find \emphas{correlations} between variables.

    \subsection{Factors}

    \define{Factors} are the variables that experimenters control during an experiment in order to determine their effect on the response variable. A factor can take on only a small number of values, which are known as factor levels. Examples of factors are brand of equipment, where the factor levels are brand A, B, and C.

    A \define{treatment} is a combination of factors that has been applied to a subject. Ex: A study with two factors - control vs drug group, and patient blood type.

    \begin{tabular}{c|c|c|c|c}
        bt/drug & A   & B   & AB    & O    \\\hline
        C       & C,A & C,B & C, AB & C, O \\
        D       & D,A & D,B & D, AB & D, O \\
    \end{tabular}

    C,A and D,A are two possible treatments.

    \subsection{Crossed vs. Nested}

    When we have two factors, the design can be either \emphas{crossed or nested}.

    A \define{crossed design} is where every possible treatment (combinations of factor levels) is present in the study.

    A \define{nested design} is where not all possible treatments are present. For example, if we have 8 schools and two teaching methods, but not all schools teach both types.

    \begin{tabular}{c|c|c|c|c|c|c|c|c}
          & A   & A   & B   & B   & C   & C   & C   & C   \\\hline
        1 & 1,A & 1,A & 1,B & 1,B &     &     &     &     \\
        2 &     &     &     &     & 2,C & 2,C & 2,C & 2,C \\
    \end{tabular}

    Here, we would say that schools are nested within class format.

    \subsection{Blocking}

    Consider an experiment that is trying to determine if a new supplement increases vitamin C absorption.

    Let Y response variable = vitamin C absorption

    Let Factor A = group with levels ``control'' and ``new''.

    There is a \emphas{total variance} in how subjects absorb vitamin C. If we can explain more of that variance, we are more likely to be able to tell if factor A had an effect.

    \define{Blocking} is using another explanatory variable to further split the subjects. For example, perhaps gender affects how subjected absorb vitamin C. Then we could first block (separate) subjects by gender, then randomly assign them to factor A. This may reduce unexplained variance in Y.

    \subsection{ANOVA Designs}

    Most ANOVA models assume an underlying structure to the data,

    \begin{align*}
        Y & = [overall \; constant] + [same \; things] \\
          & + [individual \; error]
    \end{align*}

    For example, we may say that the height of a tree has some \emphas{overall value} which could be affected by the \emphas{same things}, and then also \emphas{individual variance (error)}

    This is similar to a \emphas{regression model}

    Depending on the design of the study, we use different models.

    \define{Completely Randomized Designs} are where treatments are assigned to subjects randomly.

    For example, say we assign a sample of trees randomly to 4 different fertilizers (A,B,C,D).

    Then our model would be 

    \begin{align*}
        height & = [some \; constant] + [fertilizer \; effect] \\
                & + [ind \; error]
    \end{align*}

    \section{Single Factor ANOVA} 

    Consider Y = numeric, and X = categorical with ``a'' categories total. The basic model we use is,

    \begin{equation*}
        Y_{ij} = \mean{i} + \err{ij} \quad i=1,\dots,a, j=1,\dots,n_i
    \end{equation*}

    This is called the \define{group mean model}

    \begin{itemize}
        \item $Y_{j}$ = the jth value of Y for the ith group.
        \item $\mean{i}$ = the unknown, true population mean for the ith group.
        \item $\err{ij}$ = the jth residual/error of Y for the ith group.
    \end{itemize}

    We assume
    \begin{enumerate}
        \item $Y_{ij}'s$ were randomly sampled (independent).
        \item The ith group is independent ($i=1,\dots,a$)
        \item $\err{ij} \sim N(o, \sigvar{\epsilon})$ (errors are independent and normally distributed with mean 0, constant variance.)
    \end{enumerate}

    Notice that $Y_{ij}$ is a linear combination of the RV $\err{ij}$, so

    \begin{align*}
        \exp{Y_{ij}} & = \exp{\mean{i} + \err{ij}} \\
                    & = \exp{\mean{i}} + \exp{\err{ij}} \\
                    & = \mean{i} + 0 = \mean{i}
    \end{align*}

    \begin{align*}
        \var{Y_{ij}} & = \var{\mean{i} + \err{ij}} \\
                & = \var{\mean{i}} + \var{\err{ij}} \\
                & = 0 + \sigvar{\epsilon} = \sigvar{\epsilon}
    \end{align*}

    If we assume $\err{ij}$ is normally distributed $\implies$ $Y_{ij}$ is normally distributed. Therefore, 

    \begin{equation*}
        Y_{ij} \sim N(\mean{i}, \sigvar{\epsilon})
    \end{equation*}

    \subsection{Estimating \texorpdfstring{$\mean{i}$}{mui} and Notation}

    Estimate population mean with sample mean. Get $\mean{i}$ from $\bar{y_{i}}$.

    First, some notation. Let

    \begin{enumerate}
        \item $Y_{i\bullet} = \sum_{j=1}^{n_i} Y_{ij}=$ total for all $n_i$ observations in group i.
        \item $Y_{\bullet\bullet} = \sum_{i=1}^{a} \sum_{j=1}^{n_i} Y_{ij}=$ total for entire sample regardless of group.
        \item $n_{T} = \sum_{i=1}^a n_i =$ overall sample size regardless of group.
        \item $\bar Y_{i\bullet} = Y_{i\bullet}/n_i=$ sample mean for all observations in group i.
        \item $\bar Y_{\bullet\bullet} = \sum_{i=1}^a \sum_{j=1}^{n_i} Y_{ij}/\sum_{i=1}^a n_i = \sum_{i=1}^a \bar Y_{i\bullet} n_i / n_T =$  overall sample mean.
    \end{enumerate}

    \drawline

    Is $\bar Y_{i\bullet}$ a good estimator for $\mu$?

    Consider,

    \begin{align*}
        Q & = \sum_{i=1}^a \sum_{j=1}^{n_i} \err{ij}^2 = \text{ Sum of Squared Errors} \\
            & = \sum_i \sum_j (Y_{ij} - \mean{i})^2 \\
            & = \sum_j (Y_{ij} - \mean{1})^2 + \sum_j (Y_{ij} - \mean{2})^2  + \dots \sum_j (Y_{ij} - \mean{a})^2 
    \end{align*}

    \begin{align*}
        \frac{dQ}{d\mean{i}} & = \frac{d}{d\mean{i}} \left\{\sum_j (Y_{ij} - \mean{1})^2 +\dots+ \sum_j (Y_{ij} - \mean{i})^2  \dots \right\} \\
            & = \frac{d}{d\mean{i}} \left\{\sum_j (Y_{ij} - \mean{i})^2 \right\} \\
            & = 2\sum_j (Y_{ij} - \mean{i}) (-1) = -2 \sum_j (Y_{ij} - \mean{i})     
    \end{align*}

    then expand the sum and set to ``0''.

    \begin{align*}
        -2 \sum_j (Y_{ij} - \mean{i})  & = 0 \implies \sum_j (Y_{ij} - \mean{i}) = 0 \\
            & = \sum_j{Y_{ij}} - \sum_j{\mean{i}} = 0 \\
            & = \sum_j \mean{i} = \sum_j Y_{ij} \\
            & = n_i \mean{i} = \sum_j Y_{ij} \\
            & = \hat{\mean{i}} = \sum_j \frac{Y_{ij}}{n_i} \\
            & = \bar Y_{i\bullet}
    \end{align*}

    Thus, $\hat{\mean{i}} = \bar{Y_{i\bullet}}$ is the estimator of $\mean{i}$ that minimizes the sum of squared errors.

    \drawline

    Mean and Variance of $\hat{\mean{i}}$
    \begin{align*}
        \exp{\hat{\mean{i}}} & = \exp{\sum_j Y_{ij}/n_i} = \frac{1}{n_i} \exp{\sum_j Y_{ij}} \\
            & = \frac{1}{n_i} \sum_j \exp{Y_{ij}} = \frac{1}{n_i} \sum_j \mean{i} \\
            & = \frac{1}{n_i} (n_i \mean{i}) = \mean{i} \\
        \implies \exp{\hat{\mean{i}}} & = \mean{i} 
    \end{align*}

    \begin{align*}
        \var{\hat{\mean{i}}} & = \var{\sum_j Y_{ij}/n_i} = (\frac{1}{n_i}^2) \var{\sum_j Y_{ij}} \\
            & = \frac{1}{n_i}^2 \sum_j \var{Y_{ij}} = \frac{1}{n_i^2} n_i \var{\epsilon} \\
        \implies \var{\hat{\mean{i}}} & = \frac{\sigvar{\epsilon}}{n_i}
    \end{align*}

    Since $\hat{\mean{i}} = \sum_j Y_{ij}/n_i$ is a linear combination of normal RVs, it is normally distributed.

    Thus, 

    \begin{align*}
        \bar Y_{i\bullet} = \hat{\mean{i}} \sim N(\mean{i}, \sqrt{\sigvar{\epsilon}/n_i})
    \end{align*}

    \subsection{Residuals/Errors}

    The errors for a model are the actual values minus the estimated values, so

    \begin{equation*}
        \err{ij} = Y_{ij} - \mean{i}
    \end{equation*}

    The estimated errors (residuals) are 

    \begin{equation*}
        e_{ij} = y_{ij} - \hat{\mean{i}} \; (or \; \hat{\err{ij}} = y_{ij} - \hat{\mean{i}})
    \end{equation*}

    \subsection{Total Variance Partitioning}

    The total or overall variance of a data set is widely accepted to be the sum of squared distance from the mean. It is often denoted SSTO and defined as 

    \begin{align*}
        SSTO = \sum_i \sum_j (Y_{ij} - \bar Y_{\bullet\bullet})^2
    \end{align*}

    We can decompose $(Y_{ij} - \bar Y_{\bullet\bullet})$

    \begin{align*}
        (Y_{ij} - \bar Y_{\bullet\bullet}) & = Y_{ij} - \bar Y_{\bullet\bullet} + \bar Y_{i\bullet} - \bar Y_{i\bullet} \\
            & = (\bar Y_{i\bullet} - \bar Y_{\bullet\bullet}) + (Y_{ij} - \bar Y_{i\bullet}) \\
            & = (\bar Y_{i\bullet} - \bar Y_{\bullet\bullet}) + e_{ij}
    \end{align*}

    \newcommand{\bul}{\bullet}

    Square both sides and expand,

    \begin{align*}
        (Y_{ij} - \bar Y_{\bul\bul})^2 & = (\bar Y_{i\bul} - \bar Y_{\bul\bul})^2 + (e_{ij}^2) + 2(\bar Y_{i\bul} - \bar Y_{\bul\bul}) e_{ij} \\
    \end{align*}

    \begin{align*}
        \sum_i \sum_j (Y_{ij} - \bar Y_{\bul\bul})^2 & = \sum_i \sum_j (\bar Y_{i\bul} - \bar Y_{\bul\bul})^2  \\
        & + \sum_i \sum_j(e_{ij}^2) + 0 \\
    \end{align*}

    In other words, the total variance can be partitioned into 2 parts:

    \begin{enumerate}
        \item $SSA = \sum_i \sum_j (\bar Y_{i\bul} - \bar Y_{\bul\bul})^2 = \sum_i n_i (\bar Y_{i\bul - \bar Y_{\bul\bul}})^2$ = \define{Factor A Sum of Squares}.
        \item $SSE = \sum_i \sum_j e_{ij}^2$ = \define{Sum of Squares Error}.
    \end{enumerate}

    In general, we want SSE to be small (low error) so then SSA is large. This means that much of the variance in Y is due to a difference in the overall mean vs. group mean, and not due to error. This is why it is called analysis of variance.

    \subsection{SS Properties}

    Notice SSTO, SSA, and SSE are sums of squared values, so they keep growing larger as i or j increase. They never converge.

    Due to this, we \emphas{stabilize or standardize} the SS values to obtain \define{``Mean Square''} values. The value we stabilize them with is the df (degrees of freedom).

    \drawline

    To find $df\{SSTO\} = MSTO$

    $df\{SSTO\}$ = number of observations that can freely vary, subject to our constraint.

    Our constraint 
    \begin{align*}
        \implies & \sum_i \sum_j (Y_{ij} - \bar Y_{\bul\bul}) = 0 \\
        \implies & \bar Y_{\bul\bul} = \sum_i \sum_j Y_{ij}/n_T 
    \end{align*}

    So, we may allow $n_T - 1$ values of $Y_{ij}$ to vary freely, but the last one cannot.

    So, $df\{SSTO\} = n_T - 1$. Thus,

    \newcommand{\df}[1]{df\left\{#1\right\}}
    \begin{equation*}
        MSTO = SSTO/\df{SSTO} = \frac{SSTO}{n_T-1}
    \end{equation*}

    \drawline

    To find $\df{SSA} = MSA$

    \begin{align*}
        SSA = \sum_i n_i (\bar Y_{i\bul} - \bar Y_{\bul\bul})^2
    \end{align*}

    \newcommand{\bull}{\bullet\bullet}

    We have ``a'' values of $\bar Y_{i\bul}$ that we are summing over, and the constraint is $\sum_i n_i (\bar Y_{i\bul} - \bar Y_{\bull}) = 0$

    Thus, $\df{SSA} = a - 1$

    \begin{align*}
        MSA & = SSA/\df{SSA} \\
        \exp{MSA} & = \sigvar{\epsilon} + \left(\sum_i n_i (\mean{i} - \mu )^2\right)/(a-1)
    \end{align*}

    \drawline

    To find $\df{SSE} = MSE$

    $SSE = \sum_i \sum_j (Y_{ij} - \bar Y_{i\bul})^2$ which uses $n_T$ values of $Y_{ij}$ and has ``a'' constraints: $\sum_j (Y_{ij} - \bar Y_{i\bul}) = 0$

    Thus $\df{SSE} = n_T - a$

    \begin{align*}
        MSE & = SSE/\df{SSE} \\
        \exp{MSE} & = \sigvar{\epsilon}
    \end{align*}

    \subsection{F test for equal means}

    The first main question of single factor ANOVA is ``is the statistical evidence to suggest the group means are equal for all groups?''

    To answer that question, we will compare MSE to MSA, and use the fact that $\exp{MSE} = \exp{MSA}$ if and only if $\mean{i} = \mu$ for all i (i.e. the group means equal the overall mean).

    \drawline

    Step 1: Declare \emphas{Hypotheses}

    $H_0: \mean{1} = \mean{2} = \dots = \mean{a}$

    $H_A:$ At least one $\mean{i} \ne$ to another.

    Assume $H_0$ is true.

    \drawline
    
    Step 2: Find \emphas{test-statistic}

    Let our test-statistic be 

    \begin{equation*}
        F_s = \frac{MSA}{MSE}
    \end{equation*}

    Notice if $H_0$ were exactly true, $F^* = 1$

    We can show that, if $H_0$ is true, $F_s$ is the ratio of two independent $\chi^2$ variables, and is F distributed with $\df{num} = a-1, \df{denom}=n_T-a$

    \drawline

    Step 3: Calculate \emphas{p-value}

    \begin{align*}
        p & = \prob{F > F_s} \text{ where}\\
        \df{num} & = a-1, \df{denom} = n_T - a
    \end{align*}

    \drawline

    Step 4: \emphas{Decision Rule and Conclusion}

    If p-value $< \alpha$, reject $H_0$

    If p-value $\geq \alpha$, fail to reject $H_0$

    Note: SSE can be written as :

    \begin{align*}
        SSE = \sum_i \svar{i} (n_i - 1)
    \end{align*}
    $\svar{i} =$ sample variance for the ith group

    If we fail to reject $H_0$, that means that the tested factor has no effect on the mean.

    If we reject $H_0$, the tested factor has an effect on the mean.

    \subsection{Alternative Form of Single Factor ANOVA}

    This is called the \define{factor effect model}.

    \begin{equation}
        Y_{ij} = \bar \mu + \gamma_i + \err{ij} \text{ where } \sum^a_{i=1} \gamma_i = 0
    \end{equation}

    We have the same 

    \section{Alternative Approach to F test}

    Consider a ``full'' and ``reduced'' model,

    \define{Full Model:} The movel with the most parameters being considered.

    \define{Reduced Model:} is a subset of the full model.

    For single factor ANOVA, 

    \begin{align}
        Y_{ij} & = \mean{i} + \err{ij} \quad \text{Full: ``a'' parameters} \\
        Y_{ij} & = \mean{0} + \err{ij} \quad \text{Reduced: 1 parameter}
    \end{align}

    Step 1: Declare Hypotheses

    $H_0$: the reduced model is a statistically better fit.

    $H_A$: the reduced model is not a statistically better fit.

    \begin{align}
        SSE_F & = SSE \text{ of full model} \\
        SSE_R & = SSE \text{ of Reduced model} 
    \end{align}

    \drawline

    Step 2: Calculate F test-statistic

    \begin{equation}\label{Fstat_alt}
        F_s = \left[\frac{SSE_R - SSE_F}{\df{SSE_R} - \df{SSE_F}}\right] \Big/ \left[\frac{SSE_F}{\df{SSE_F}}\right]
    \end{equation}

    where 

    \begin{align}
        \df{num} & = \df{SSE_R} - \df{SSE_F} \\
        \df{denom} & = \df{SSE_F} \\
        \df{SSE} & = n_T - a \\
        \df{SSE_F} & = n_T - a \\
        \df{SSE_R} & = n_T - 1 
    \end{align}

    NOTE: In \emphas{Single Factor ANOVA}, (\ref{Fstat_alt}) reduces to 

    \begin{equation*}
        F_S = \frac{MSA}{MSE}
    \end{equation*}

    \section{Calculating Power}

    The \define{power} of a test is the probability of rejecting $H_0$ when $H_0$ is false. Aka the probability of correctly rejecting the null.

    \begin{equation}
        Power = \prob{\text{Reject } H_0 | H_0 \text{ false}}
    \end{equation}

    This calculation requires that we assume a specific $H_A$ is true, i.e that we assume some values of $\mean{i}$ are true, where not all $\mean{i}$ are equal.

    Then, $F_s$ is no longer F distributed, it is non-central-F, with parameter $\phi$ (phi) which measures how much different $F_s$ is under $H_A$ to the under $H_0$.

    \begin{equation}
        \phi = \frac{1}{\sigdev{\epsilon}} \sqrt{\frac{\sum^n_{i=1} n_i (\mean{i} - \mean{0})^2}{a}}
    \end{equation}

    In practice, $\mean{i}$ and $\mean{0}$ must be approximated based on sample values.

    After calculating $\phi$, use the \define{power table} to find power values given $\df{num}, \df{denom}, \alpha, \phi$

    For the Power table, \emph{round down} all degrees of freedom, but if $\phi$ isn't in the table, round up or down to pick the closest value.

    NOTE: $\sqrt{MSE}$ is an estimate for $\sigdev{\epsilon}$

    NOTE: Power calculations require strict assumptions because we assume $\mean{i}, \mean{0}, \sigdev{\epsilon}$. Be sure to take power calculations as estimates.

    \section{Sample Size Calculations}

    A common request is to determine the minimum sample size needed to achieve a particular power for a given $\alpha$. 

    Make a few changes,

    \begin{enumerate}
        \item We assume all $n_i$ values will be equal. Call this common value $n_c$.
        \item We still assume $\sigvar{\epsilon}$ is known.
        \item Instead of specifying $\phi$, focus on $\Delta$.
    \end{enumerate}

    \begin{equation}
        \Delta = \text{Largest effect size} = max\{\mean{i}\} - min\{\mean{i}\}
    \end{equation}

    and more specifically,

    $\Delta/\sigvar{\epsilon}$ = the number of standard deviations of $\err{ij}$ the largest mean is suspected to be from the smallest.

    $\beta = \prob{\text{Type II Error}}$

    $Power = 1 - \beta$

    To use the table, select the table section witht the desired Power, then choose the column with the closest value for $\Delta/\sigma$ (round up or down), and then choose the subcolumn with the desired $\alpha$. Go down to the row with the correct $a$ value. The value in that slot is the minimum sample size per group $n_c$. 

    Notes:
    \begin{itemize}
        \item Generally, as $\alpha \uparrow, \beta \downarrow$ and vice versa.
        \item The larger $\phi$, the more difference there was in the assumed $\mean{i}$, and the higher the power.
        \item The larger $\Delta/\sigdev{\epsilon}$, the less $n_c$ has to be for a particular power.
    \end{itemize}

    \section{Confidence Intervals for SFA}

    If we have rejected the null hypothesis that the population means are equal, confidence intervals answer the followup question, ``which means are different?''

    \subsection{CI for Single mean}

    For estimating a particular $\mean{i}$, we can use the fact that 

    \begin{align}
        \bar{Y_{i\bul}} & \sim N(\mean{i}, \sqrt{\sigvar{\epsilon}/n_i}) \\
        \implies \frac{\bar{Y_{i\bul}} - \mean{i}}{\sqrt{\sigvar{\epsilon}}/n_i} & \sim N(0,1)
    \end{align}

    $\sigvar{\epsilon}$ is unknown but can be estimated with $\sqrt{MSE}$. However, then $(\bar{Y_{i\bul}} - \mean{i})/\sqrt{MSE/n_i}$ follows a t distribution with $df = n_T - a$.

    Thus a $(1-\alpha) 100\%$ CI for $\mean{i}$ is:

    \begin{equation}
        \bar{Y_{i\bul}} \pm t_{1 - \alpha/2; n_T - a} \sqrt{MSE/n_i}
    \end{equation}

    \subsection{CI for difference in Means}

    For a CI comparing $\mean{i}$ to $\mean{i}'$ where $i \ne i'$, we have a similar conclusion that,

    \begin{equation}
        (\bar{Y_{i\bul}} - \bar{Y_{i\bul}'}) \sim N(\mean{i} - \mean{i}', \sqrt{\sigvar{\epsilon} \left(\frac{1}{n_i} + \frac{1}{n_i'}\right)}
    \end{equation}

    and so 

    \begin{equation}
        \frac{[(\bar{Y_{i\bul}} - \bar{Y_{i\bul}'}) - (\mean{i} - \mean{i}')]}{\sqrt{\sigvar{\epsilon}\left(\frac{1}{n_i} + \frac{1}{n_i'}\right)}} \sim t_{n_T - a}
    \end{equation}

    NOTE: The abova results depend on the assumptions of SFA.

    Thus, a $(1-\alpha)100\%$ CI for $(\mean{i} - \mean{i}')$ is 

    \begin{equation}
        (\bar{Y_{i\bul}} - \bar{Y_{i\bul}'}) \pm t_{1-\alpha/2; n_T - a} \sqrt{MSE\left(\frac{1}{n_i} + \frac{1}{n_i'}\right)}
    \end{equation}

    \subsection{CI for contrast of means}

    A \define{contrast} of group means is a linear combination of $\mean{i}$ where the constants in front of the $\mean{i}$'s sum to 0.

    \begin{equation}
        \sum^a_{i=1} c_i \mean{i} \text{ where } \sum_i c_i = 0
    \end{equation}

    Notice that $\sum_i c_i \bar{Y_{i\bul}} \sim N(\sum c_i \mean{i}, \sqrt{MSE \sum c_i^2/n_i})$

    Thus a $(1-\alpha)100\%$ CI for $\sum c_i \mean{i}$ is 

    \begin{equation}
        \sum_i c_i \bar{Y_{i\bul}} \pm t_{1-\alpha/2; n_t - a} \sqrt{MSE \sum c_i^2/n_i}
    \end{equation}

    Notice: all CIs use $\sqrt{MSE}$, which works because the assumption of SFA is that all group std. dev.'s are equal.

    \subsection{Simultaneous CI}

    We often want to make multiple confidence intervals for an ANOVA model. If each CI has $\alpha = \prob{\text{Type 1 Error}}$, the error builds with each interval. So for a simultaneous CI with 2 intervals, the maximum error is 1.0.

    Proof: Let $A = \#$ of Type I errors made out of 2 intervals.

    \begin{align*}
        \prob{A \ge 1} & = 1 - \prob{A < 1} = 1 - \prob{A = 0} \\
                        & = 1 - (1 - \alpha)^2 = 0.19
    \end{align*}

    This can be generalized to: for g confidence intervals at level $(1-\alpha)100\%$, the overall/family/simultaneous error rate is: $1-(1-\alpha)^g$

    \subsection{Correcting CI to reduce alpha}

    $(1-\alpha)100\%$ CIs that correct for a increased $\alpha$
    \begin{align}
        \text{Tukey's} & \quad \bar{Y_{i\bul}} - \bar{Y_{i\bul}'} \pm T \sqrt{MSE/(n_i + n_i')} \\
        \text{Scheffe's} & \quad \sum_i c_i \bar{Y_{i\bul}} \pm S \sqrt{MSE \left(\sum_i c_i/n_i\right)} \\
        \text{Bonferroni's} & \quad \bar{Y_{i\bul}} - \bar{Y_{i\bul}'} \pm B \sqrt{MSE/(n_i + n_i')}
    \end{align}

    \drawline

    Multipliers

    \begin{align}
        T & = \frac{1}{\sqrt{2}} q_{1-\alpha; a, n_t - a} \\
        S & = \sqrt{(a-1) F_{1-\alpha; a - 1, n_T - a}} \\
        B & = t_{1 - \alpha/(2g); n_T - a} 
    \end{align}

    \drawline
    
    Distributions
    \begin{itemize}
        \item $q(1-\alpha; a, n_t - a)$ represents the $(1-\alpha)100^{th}$ percentile of a studentized range distribution at $1-\alpha$ with degrees of freedom $a$ and $n_T - a$.
        \item $F(1-\alpha; a-1, n_T - a)$ represents the test statistic from an F distribution at $(1-\alpha)$ given $df_1 = a-1$ and $df_2 = n_T - a)$.
        \item $t(1-\alpha(2g); n_T - a)$ represents a test statistic from a t distribution at $1-\alpha/(2g)$ with degrees of freedom $n_T - a$. Note that $1-\alpha/2$ becomes over $2g$, this is because Bonferroni simply decreases the $\alpha$ value per interval to sum to $\alpha$ overall.
    \end{itemize}

    \drawline

    Applications
    \begin{itemize}
        \item Tukey results in narrower CIs when only pairwise differences are being considered.
        \item Scheffe is best for taking multiple contrasts, other than pairwise differences (or as a combination of pairwise differences and general contrasts). In general, if an interval is only pairwise combinations, Tukey is best.
        \item Bonferroni is best used when a relatively small number of intervals is being taken (g is small). For large values of g, the error becomes so small that the actual multiplier becomes very large, and thus the interval widens.
    \end{itemize}

    NOTE: There is a change in the conclusion of these three corrected CIs, we say we are \define{overall} (or family wise or simultaneously) $(1-\alpha)100\%$ confident that $\dots$.

\end{multicols}
\end{document}